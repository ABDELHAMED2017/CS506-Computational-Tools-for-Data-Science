\documentclass[11pt]{article}
\pagestyle{empty}

\setlength{\textheight}{8.5in}
\setlength{\topmargin}{0.5in}
\setlength{\headheight}{0in}
\setlength{\headsep}{0in}
% %% \setlength{\footheight}{0in}
\setlength{\oddsidemargin}{0in}
\setlength{\textwidth}{6.5in}

\usepackage{times}
\usepackage{url}
\usepackage{algorithm}
\usepackage{mathtools}
\usepackage{mathptmx}
\usepackage{amssymb}
\usepackage{color}

\begin{document}
\sloppy 
\begin{center}
\LARGE CAS CS 505\\
\Large Computational Tools for Data Science\\
\Large\rm Fall 2016\\~\\
\end{center}

\noindent{\large\bf Meeting Place:} CAS 325\\[\baselineskip]
\noindent{\large\bf Meeting Time:} TR 11-12:30
\\[\baselineskip] 

\noindent{\large\bf Instructor:} Prof.\ Mark Crovella\\[0.75\baselineskip]
\begin{minipage}[t]{0.60\textwidth}
\begin{itemize}
\item {\bf Office:} MCS-140E
\item {\bf Office Hours:} {\small M 2-3:30, R 3-4:30}
\item {\bf Email:} crovella@bu.edu
\end{itemize}
\end{minipage}
~\\~\\~\\~\\
 \noindent{\large\bf Teaching Fellow:} Ms.\ Katherine Missimer\\[0.75\baselineskip]
 \begin{minipage}[t]{0.60\textwidth}
 \begin{itemize}
 \item {\bf Office Hours:} {\small W 4-5:30, F 5-6:30}
 \item {\bf Office Hours Location:} Undergrad Lab, EMA 302
 \item {\bf Lab Tutoring Hours:} {\small F 3-5.}
 \item {\bf Email:} kzhao@bu.edu
 \end{itemize}
 \end{minipage}

\section*{Overview of the Course}

This course is targeted at students who require a basic level of
proficiency in working with and analyzing data.  The course emphasizes
practical skills in working with data, while introducing students to a
wide range of techniques that are commonly used in the analysis of data,
such as clustering, classification, regression, and network analysis.
The goal of the class is to provide to students a hands-on understanding
of classical data analysis techniques and to develop proficiency in
applying these techniques in a modern programming language (Python). 

Broadly speaking, the course breaks down into three main components,
which we will take in order of increasing complication:  (a)
unsupervised methods; (b) supervised methods; and (c) methods for
structured data.

Lectures will present the fundamentals of each technique; focus is not
on the theoretical underpinnings of the methods, but rather on helping
students understand the practical settings in which these methods are
useful.  Class discussion will study use cases and will go over relevant
Python packages that will enable the students to perform hands-on
experiments with their data. 

Prerequisites: Students taking this class must have some prior familiarity with
programming, at the level of CS 105, 108, or 111, or equivalent.   CS
132 or equivalent (MA 242, MA 442) is required.  CS 112 is also helpful.

\section*{Readings} 

There is no text.   Lecture notes will be posted online.

Some recommended texts are:
\begin{enumerate}
\item Python for Data Analysis
  (http://shop.oreilly.com/product/0636920023784.do)
\item Programming Collective Intelligence (http://shop.oreilly.com/product/9780596529321.do)
\end{enumerate}

\section*{Web Resources} 

The slides I use are actually executable python scripts, using the
\texttt{ipython notebook.}   If you have ipython notebook, you can
download and execute the examples on your own computer, and you can
modify them any way you'd like, play around with them, experiment, etc.

The slides I use in lecture are published on \texttt{github.}   The
repository is
\url{https://github.com/mcrovella/CS505-Data-Science-in-Python}.  If you want
to access the repository using \texttt{git,} please feel free but you
can simply download directly from the web site if you prefer.
 
\section*{Homeworks and Project}

\begin{enumerate}
\item There will nine homework assignments.  In a typical 
assignment you will 
analyze one or more datasets using the tools and techniques presented in
class.

\item In addition, there will be a final project.  For the project you
  will extract some
knowledge or conclusions from the analysis of dataset of your choice. The analysis
will be done using a subset of the methods we described in class.  The
final project will require a proposal, two progress reports, and a final
presentation in poster form.

The project will have three essential components: 1) a data collection
piece (which may involve crawling or calls to an API, combining data
from different sources etc), 2) a data analysis piece (which will
involve applying different techniques we described in class for the
analysis) and 3) a conclusion component (where the results of the data
analysis will be drawn).  The students will submit a 5-page report
explaining clearly all the three components of their project. Finally a
poster presentation will be required where the students will be prepare
to present their effort and results in front of their poster. 

As an example, you may choose to collect data from Twitter related
to a specific topic (e.g., Ebola virus) and then measure the intensity
of posts about a topic in different areas of the world etc.  Other
examples of projects may include (but are not limited to): analysis of
MBTA data, analysis of NYC data, crawling of YouTube (or other social
media data) and analysis of social behavior like trolling, bullying
etc. 

The project is due by the last day of class (December 8). The project presentations will be
given in the form of a final poster explaining components 1, 2 and 3 of
the project. 

You are expected to work individually on homeworks and on the final
project.  There will be no final exam.   

\item Homeworks will be submitted via \texttt{github}.
\end{enumerate}

\section*{Piazza}

We will be using Piazza for class discussion. The system is really well
tuned to getting you help fast and efficiently from classmates, Ms.\ Missimer,
and myself. Rather than emailing questions to the teaching staff,
I encourage you to post your questions on Piazza.   Our class Piazza
page  is at: \url{https://piazza.com/bu/???}. 
We will also use Piazza for distributing materials
such as homeworks and solutions.

When someone posts a question on Piazza, if you know the answer, please
go ahead and post it.   However pleased \emph{don't} provide answers to homework
questions on Piazza.   It's OK to tell people \emph{where to look} to
get answers, or to correct mistakes;  just don't provide actual solutions
to homeworks.

\section*{Programming Environment}

We will use \texttt{python} as the language for teaching and for
assignments that require coding.    Instructions for installing and
using Python are on Piazza.

% \section*{Clickers}

% We will be using ``Peer instruction'' as part of the lectures.  This
% requires you to answer occasional questions during lecture, sometimes
% after discussion with your classmates.   

% To support this, we will use clickers for student feedback during lecture.  If you
% don't already have one, you need to get one.  It is the Turning
% Technologies Response Card RF (ISBN 9781934931394).  This is the
% standard ``BU clicker'' and you can get it at
% the BU Bookstore.   You will need to bring it to every lecture.
% % register it on Blackboard 

% Also, to encourage this interactive style of lecture, I will ask you to
% put away laptops and phones during lecture.    Please note that I will
% post copies of lecture slides online for you to consult after class if
% you like.

% You will also want to bring pencil and paper to lecture.   This isn't
% absolutely critical, but you will find it easier if you can jot a note
% or two while responding to clicker questions.

\section*{Course and Grading Administration}

Assignments will be submitted using \texttt{github}.   Ms.\ Missimer will
explain how to submit assignments.  

% \emph{NOTE: IMPORTANT:} Late assignments \textbf{WILL NOT} be accepted.   However, your final
% grade will be based on the top 10 homeworks submitted (out of 12).   

Final grades will be computed based on the following:
\begin{description}
\item[50\%] Homework assignments.  
% \item[5\%] Attendance and In-class participation via clicker.
% \item[20\%] Midterm
\item[50\%] Final Project
\end{description}

% To get full credit for class participation by clicker, you need to use
% the clicker on 85\% of the questions that are posed in lecture.   So if
% you miss a question here or there, or forget your clicker one day, don't
% worry as long as you come to lecture consistently.

The exact cutoffs for final grades will be determined after the class is
complete.

\newpage

\section*{Academic Honesty}

You may discuss homework assignments with classmates, but you are 
solely responsible for what you turn in. Collaboration in the form of
discussion is allowed, but all forms of cheating (copying parts of a
classmate's assignment, plagiarism from books or old posted solutions)
are NOT allowed. We -- both teaching staff and students -- are expected
to abide by the guidelines and rules of the Academic Code of Conduct
(which is at
\url{http://www.bu.edu/dos/policies/student-responsibilities/}).

You can probably, if you try hard enough, find solutions for homework
problems online.    Given the nature of the Internet, this is
inevitable.   Let me make a couple of comments about that:
\begin{enumerate}
\item If you are looking online for an answer because you don't know how
  to start thinking about a problem, talk to Ms.\ Missimer or myself, who may be
  able to give you pointers to get you started.  Piazza is great for
  this -- you can usually get an answer in an hour if not a few minutes.
\item If you are looking online for an answer because you want to see if
  your solution is correct, ask yourself if there is some way to verify
  the solution yourself.   Usually, there is.  You will understand what you have done
  \emph{much} better if you do that.
So ... it would be better to simply submit what you have at the deadline
(without going online to cheat) and plan to allocate more time for
homeworks in the future.
\end{enumerate}

\newpage
\section*{Course Schedule}

\small
\begin{centering}
\begin{tabular}{||l|p{3in}|l|l|l||}
\hline\hline
Date & Topics  & Reading & Assigned & Due  \\
\hline\hline
% complete basics of python
9/6 & Introduction to Python &  &  HW 0 & \\
% more sophisticated; prep for homewok 0
% get pandas and more from 
% http://twiecki.github.io/blog/2014/11/18/python-for-data-science/ 
% also cover APIs - can use some material from CS211 ?
% also cover basic visualization
9/8 & Git, Github, Python notebooks, Pandas & & & \\

\hline
% doesnt exist yet;  look at textbooks
9/13 & Probability and Statistics Refresher & &  & HW 0 \\
% doesnt exist yet;  also should develop a homework for this material
% h/w could use APIs and do simple statistics on the retrieved data.
9/15 & Linear Algebra Refresher & & & \\
\hline
% good material but need to integrate PDF and notebook
% needs some additional examples too
 9/20 & Numpy, Scikit-learn, Distance and Similarity Functions & &  & \\
% & Data Exploration & & & \\
% current slides not really usable b/c they focus on dynamic time
% warping.  some OK points, but needs a re-do based on books
9/22 & Intro to Timeseries & & HW 1.1& \\ 
% Homework 1 is OK here. - distance functions, APIs, timeseries.
\hline

% this is a set of slides in PDF.  good basics of clustering.  good
% coverage of k-means and k-means++ - just need to convert to notebook
9/27 & Clustering, k-means & & & \\
% this one is excellent - set in context of sklearn - practical and
% includes examples
9/29 & Clustering II & &HW 1.2  & \\ 
9/30 &&&& HW 1.1 \\
\hline

% pretty good PDF just need to convert to notebook
10/4 & Hierarchical Clustering & & & \\  
% EM and GMM; short but good and its already started as a notebook
10/6 & Expectation Maximization and GMM & & HW 2.1, 2.2 & \\ 
% Homework 2 depends on k-means, hierarchical, and GMM clustering 
10/7 &&&& HW 1.2 \\
\hline

10/11 &  NO CLASS; Monday Schedule &  & &  \\   
% basically in OK shape; already a notebook, needs a smoothing pass
10/13 & DB Clustering and Comparing Clustering Algorithms & & & \\
10/7 &&&& HW 2.1 \\
\hline

% OK but key material is in PDF.  needs conversion to notebook
% existing notebook is mainly a sketch of next lecture
10/18 & Dimensionality Reduction - SVD I &&&\\
%  there is a lot of good basic material here -- TF/IDF and SVD
% but needs LOTS of expansion -- explain ideas behind SVD (not currently
% there)
%
% also there is a guest lecture here by Davide on web scraping
% need to figure out a replacement for him! (this is good stuff and
% could be much earlier in course)
% HW 3.1 is scraping, 3.2 is regression
10/20 & SVD II and Web Scraping & & HW 3.1, 3.2 & \\ 
10/21 &&&& HW 2.2 \\

\hline

% should we hold a midterm?  I am skeptical.
10/25 & Open & &  & \\ 
10/27 & Classification: Decision Trees & & & \\ 
10/28 &&&& HW 3.1 \\
\hline

11/1 & Classification: SVM, Naive Bayes &  &  &\\ 
11/3 & Regression: Linear Regression & & &\\ 
11/4 &&&& Proj Proposal\\
\hline

11/8 & Logistic Regression & & &\\ 
% homework 3 OK here; scraping and logistic regression
11/10 & Linear Regression II & & & \\ 
% progr report here is "how much have you scraped?"
11/11 &&&& Prog Report 1\\
\hline

11/15 & Recommendation Systems & & &\\ 
11/17 & Network Analysis I & & HW 4 & \\
% HW 3.2 is regression
11/18 &&&& HW 3.2\\
\hline

% This prog report is the same as the single one last time
11/22 & Network Analysis II &  & &Prog report 2\\ 
11/24 & NO CLASS; Thanksgiving Break  & & & \\ 
 \hline

% HW 5 is on map reduce; requires triangle counting
11/29 & Graph Clustering &  & & \\ 
12/1 & Text Analysis and Topic Modeling &&HW 5 &\\
12/2 &&&& HW 4\\
\hline 

12/6 & Wrapup & & & \\ 
12/8 & Poster Session &  &  &\\ 
\hline 
12/12 &&&& HW 5\\

\hline\hline


\end{tabular}\\
\end{centering}


\end{document}
